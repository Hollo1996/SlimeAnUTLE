%----------------------------------------------------------------------------
\chapter{Összefoglalás és jövőbeli tervek}
\label{sec:conclusion}
%----------------------------------------------------------------------------


Szakdolgozatom keretében kifejlesztettem a Slime nyelvet, ami más nyelvek kiegészítésére való.
Célja, hogy az alárendelt nyelv kódja tömörebb, átláthatóbb és struktúráltabb legyen.
Ezt elsősorban template-eléssel valósítja meg, de más alapvető funkciókat is megvalósít.
Behozza a nyelvekbe az importálást, típusos változó deklarációt, típusdeklarációt, tároló típusokat, iterációt stb.
Ezen kívül számos egyedi megoldást is bevezet, mint például a "referencia" vagy a több nevű változók.
Előbbi egy hivatkozás, ami mindig kilistázza a rá illeszkedő változókat, ez valósítja meg a témakiírás egyik fő tételét is.

A kiírás szerint az elsődleges cél az IRTG nyelvtan kiegészítése volt.
A szakdolgozatban részletesen írok ennek a nyelvtannak a szakmai hátteréről és működéséről is.
Ez alatt ki is elemeztem a nyelvtan és futtatókörnyezete kapcsán felmerülő problémákat.
Úgy vélem, hogy ezek többségét maguk a fejlesztők is könnyen orvosolhatják hosszú távon, 
ezért bővítettem a feladat fókuszát és álltam elő egy univerzális megoldással.

A Slime ugyan akár 66\%-os sor-redukciót is képes elérni az IRTG-k általunk kezelt típusa esetén, még így is sok benne a redundancia.
A működés sincs kellő hatékonysággal tesztelve, éppen ezért a következő iterációkban három célt fogok szem előtt tartani.
Az első a letisztultság növelése.
Törölni fogom a kizárólag nevek kezelésére használatos Name osztályt és elő fogok állni egy sokkal kézenfekfőbb megoldással, ami a neveket is Text-ként kezeli.
Lehetőséget fogok rá biztosítani, hogy Slot-ba szúrás esetén a Slot megszűnjön.
Létre fogok hozni egy ősosztályt a Cont változók egységes kezelésére.
Ezt követőena praktikus kódolásra összpontosítok majd.
Opcionálissá fogom tenni a legtöbb, csupán a tagolás átláthatóságát segítő jelölést.
Lehetővé fogom tenni a műveletek láncolását és az operátor zárójelek összevonását.
Ezeket követően kerül majd sor a teljeskörű validációra.
A cél a 100\%-os tesztlefedettség mind kód, mind funkcionalitás tekintetében.
Ekkor várhatóan már 85\%-os sorredukcióra is képes lesz, és az átláthatoság is látványosan javul majd. Végleges formájában a Slime egy olyan nyelv lesz, ami mindennemű felesleges redundanciát hatékonyan semlegesít. A legfontosabb cél az, hogy mindezt átláthatóan, egyszerűen és hatékonyan tegye.