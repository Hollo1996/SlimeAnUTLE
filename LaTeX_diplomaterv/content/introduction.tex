%----------------------------------------------------------------------------
\chapter{Bevezetés}
\label{sec:introducton}
%----------------------------------------------------------------------------


A természetes nyelvi technológiák legmagasabb szintjén álló olyan komplex feladatok számára, mint például a gépi fordítás, a kérdés-megválaszolás, vagy a dialógusrendszerek (chatbotok), a legnagyobb korlátot a mély szemantikai elemzés hiánya jelenti. A túlnyomórészt neurális hálók felügyelt tanulásán alapuló state-of-the-art rendszerek a nyelv szavainak jelentését olyan sokdimenziós vektorterekben reprezentálják, amelyek struktúráját nem vagy csak nagyon korlátozott mértékben ismerjük. Ezzel a megközelítéssel állnak szemben a fogalmi hálózatokon alapuló szemantikai reprezentációk, amelyek lehetővé teszik, hogy explicitebb módon vizsgáljuk a szemantikai elemzésen alapuló feladatokat, bár automatikus módszerekkel korlátozott minőségben lehet csak őket előállítani.

A mély szemantikai elemzést is segítő olyan technológiák, mint például a függőségi elemzés (dependency parsing) irányított gráfként ábrázolják a mondat szerkezeti összefüggéseit, így egy-egy szemantikai elemző pipeline alapulhat kizárólag gráf-transzformációs műveleteken: ha a nyelvi
jelentést fogalmak irányított gráfjaival reprezentáljuk, ezeket pedig a mondat szintaktikai
szerkezetét reprezentáló gráfokból kell előállítanunk, akkor a teljes feladat egyetlen komplex
gráf-transzformációként definiálható.

Az ilyen transzformációk leírására többféle formalizmus is létezik, mint például a HRG (hyperedge-replacement grammar) vagy az IRTG (interpreted regular tree grammars). Jelenleg is egy ezekkel kapcsolatos kutatás folyik az Automatizálási és Alkalmazott Informatikai Tanszéken.

A kutatás során az Alto (Algebraic Language Toolkit) nevű open-source parszerrel dolgoztunk, ami a jelenlegi leghatékonyabb környezet IRTG-k futtatására. Ugyanakkor a kutatásnak állandó gátját jelenti, hogy az IRTG nehezen átlátható, és az Alto-ból is hiányoznak fontos funkcionalitások. A problémán sokat enyhítene, ha az IRTG szabályok megengednék a reguláris kifejezések alkalmazását.

Szakdolgozatom keretében egy template-elésre alkalmas nyelvet fejlesztettem ki, ami a Slime fantázianévre hallgat. Segítségével az IRTG nyelvtanokat tömörebben és átláthatóbban lehet definiálni. Mivel az Alto Java-ban készül, a nyelvet Kotlinban készítettem ANTLRv4 segítségével. Még nincs teljesen kifejlődve, de a feladathoz szükséges megoldásokat tartalmazza. Ilyen például többek között a template-definiálás, egymásba ágyazás, RegEx-szel hivatkozás. Teljes formájában egy univerzális bővítmény lesz, ami bármely nyelv vagy szöveg felett használható.
 	 	 	
A dolgozat a következőképpen épül fel: A 2. fejezetben röviden bevezetem a téma nyelvészeti vonatkozásait és a projektet, ami kapcsán készül. Magát a nyelvet a 3. fejezetben mutatom be: ismertetem a tervezés szempontjait, amiken a design-döntések többsége nyugszik, bemutatom a nyelv részletes működését és szintaxisát, az implementáció mögötti technológiákat és megoldásokat nagy vonalakban, a nyelv fejlesztésének eddigi és ez utáni ütemtervével együtt, végül példákkal szolgálok a Slime felhasználására több alárendelt nyelvvel is.