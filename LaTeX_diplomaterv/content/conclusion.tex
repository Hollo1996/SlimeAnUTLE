%----------------------------------------------------------------------------
\chapter{Összefoglalás}
\label{sec:conclusion}
%----------------------------------------------------------------------------


Szakdolgozatom keretében kifejlesztettem a Slime nyelvet, ami más nyelvek kiegészítésére való.
Célja, hogy az alárendelt nyelv kódja tömörebb és átláthatóbb és struktúráltabb legyen.
Ezt első sorban template-léssel valósítja meg, de más alapvető funkciókat is megvalósít.
Behozza a nyelvekbe az importálást, típusos változó deklarációt, típus deklarációt, tároló típusokat, iterációt stb. .
Ezen kívül számos egyedi megoldást is bevezet, mint például a "referencia" vagy a több nevő változók.
Előbbi egy hivatkozás, ami mindig kilistázza a rá illeszkedő változókat.
Ez valósítja meg a témakiírás egyik főtételét is.

A kiírás szerint az elsődleges cél az IRTG nyelvtan kiegészítése volt.
A szakdolgozatban részletes írok ennek a nyelvtannak a szakmai hátteréről és működéséről is.
Ez alatt ki is elemeztem a nyelvtan és futtató környezete kapcsán felmerülő problémákat.
Úgy vélem, hogy ezek többségét maguk a fejlesztők is könnyen orvosolhatják hosszú távon, 
ezért bővítettem a feladat fókuszát és álltam elő egy univerzális megoldással.

A Slime ugyan akár 66\%-os sor redukciót is képes elérni az IRTG-k általunk kezelt típusa esetén, még így is sok benne a redundancia.
A működés sincs kellő hatékonysággal tesztelve.
Éppen ezért a következő iterációkban három célt fogok szem előtt tartani.
Az első a letisztultság növelése.
Törölni fogom a kizárólag nevek kezelésére használatos Name osztályt és elő fogok állni egy sokkal kézenfekfőbb megoldással, ami a neveket is Text ként kezeli.
Lehetőséget fogok rá biztosítani, hogy Slot-ba szúrás esetén a Slot megszűnjön.
Létre fogokhozni egy ősosztályt a Cont változók egységes kezelésére.
Ezt követően fogok a praktikus kódolásra összpontosítani.
Opcionálissá fogom tenni a legtöbb csupán a tagolás átláthatóságát segítő jelölést.
Lehetővé fogom tenni a műveletek láncolását és az operátor zárójelek összevonását.
Ezeket követően kerül majd sor a teljes körű validációra.
A cél a 100\%-os teszt lefedettség mint kód mint funkcionalitás tekintetében.
Ekkor várhatóan már 85\%-os sorredukcióra is képes lesz és az átláthatoság is látványosan fog javulni.

Ezeket követően beszélhetünk majd a nyelv első realise verziójáról.
A szakdolgozatban részletesen leírom a feljebb említett mérföldköveket és számos a jövőre vonatkozó ötletet.
Végleges formájában a Slime egy olyan nyelv lesz, ami minden nemű felesleges redundanciát hatékonyan semlegesít.
Legfontosabb cél az, hogy mindezt átláthatóan, egyszerűen és hatékonyan tegye.